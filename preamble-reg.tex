
\usepackage{xcolor}
\usepackage{amssymb} % For additional symbols

% Definitions for all colors used in the presentation
\definecolor{balancedOrange}{RGB}{255,216,180}
\definecolor{secondaryOrange}{RGB}{251,188,5}
\definecolor{softBlue}{RGB}{100,124,156}
\definecolor{titleColor}{RGB}{0,51,102}
\definecolor{accentColor}{RGB}{112,173,71}

    % Applying colors to Beamer elements
    \setbeamercolor{palette primary}{bg=balancedOrange,fg=titleColor}
    \setbeamercolor{palette secondary}{bg=secondaryOrange,fg=white}
    \setbeamercolor{palette tertiary}{bg=softBlue,fg=white}
    \setbeamercolor{titlelike}{parent=palette primary, fg=titleColor}
    \setbeamercolor{structure}{fg=softBlue}
    \setbeamercolor{block title}{bg=balancedOrange,fg=titleColor}
    \setbeamercolor{block body}{bg=balancedOrange!10,fg=black}

% Customizing bullet points to a smaller sideways triangle and adjusting vertical position
\setbeamertemplate{itemize item}{\raisebox{1.5pt}{\scalebox{0.6}{$\blacktriangleright$}}}
\setbeamertemplate{itemize subitem}{\raisebox{1.3pt}{\scalebox{0.5}{$\blacktriangleright$}}}
\setbeamertemplate{itemize subsubitem}{\raisebox{1.1pt}{\scalebox{0.4}{$\blacktriangleright$}}}

% Suppressing the footline on the title slide
\setbeamertemplate{footline}{
    \ifnum\insertpagenumber=1
    \else
        \hfill\insertframenumber/\inserttotalframenumber\hspace*{1em}\vspace*{1em}
    \fi
}

\newcommand{\cln}{{:\,}} %Better colon spacing in math
\newcommand{\ex}{{\rm E}}
\newcommand{\var}{{\rm Var}}
\newcommand{\cov}{{\rm Cov}}
\newcommand{\cor}{{\rm Cor}}
\newcommand{\ba}{\boldsymbol a}
\newcommand{\bb}{\boldsymbol b}
\newcommand{\cc}{\boldsymbol c}
\newcommand{\bd}{\boldsymbol d}
\newcommand{\e}{\boldsymbol e}
\newcommand{\bu}{\boldsymbol u}
\newcommand{\bv}{\boldsymbol v}
\newcommand{\x}{\boldsymbol x}
\newcommand{\y}{\boldsymbol y}
\newcommand{\A}{\boldsymbol A}
\newcommand{\B}{\boldsymbol B}
\newcommand{\C}{\boldsymbol C}
\newcommand{\D}{\boldsymbol D}
\newcommand{\E}{\boldsymbol E}
\newcommand{\bH}{\boldsymbol H}
\newcommand{\I}{\boldsymbol I}
\newcommand{\M}{\boldsymbol M}
\newcommand{\bS}{\boldsymbol S}
\newcommand{\U}{\boldsymbol U}
\newcommand{\V}{\boldsymbol V}
\newcommand{\W}{\boldsymbol W}
\newcommand{\X}{\boldsymbol X}
\newcommand{\Y}{\boldsymbol Y}
\newcommand{\nul}{\boldsymbol 0}
\newcommand{\een}{\boldsymbol 1}
\newcommand{\Bet}{\boldsymbol \beta}
\newcommand{\Gam}{\boldsymbol \gamma}
\newcommand{\Eps}{\boldsymbol \varepsilon}
\newcommand{\Sig}{\boldsymbol \Sigma}
\newcommand{\Thet}{\boldsymbol \theta}
\newcommand{\Mu}{\boldsymbol \mu}
\newcommand{\Del}{\boldsymbol \Delta}
\newcommand{\Lam}{\boldsymbol \Lambda}
\newcommand{\Tau}{\boldsymbol \tau}
